\documentclass{../../miPlantilla}

\renewcommand{\miAsignatura}{Sistemas de Big Data}
\renewcommand{\tituloTrabajo}{Eligiendo la mejor herramienta}
\renewcommand{\imagenPortada}{portada.png}

\usepackage{bookmark}

\begin{document}

\maketitle

\section{Investigar herramientas}
En este apartado vamos a investigar diferentes herramientas para análisis y visualización de datos. Nos centraremos en tres tipos principales:
\begin{itemize}
    \item Herramientas estadísticas de código abierto (Sofa Statistics, PSPP, JASP).
    \item Librerías de programación para visualización (Pandas, Matplotlib, Plotly, R).
    \item Herramientas online de visualización (Infogram, Piktochart, Flourish, RawGraphs, Datawrapper, Canva).
\end{itemize}

\subsection{Herramientas estadísticas de código abierto}

\subsubsection{SOFA Statistics}
Se trata de un software de análisis estadístico sencillo con interfaz gráfica. Nos permite realizar análisis descriptivos,
pruebas básicas y generar gráficos de barras, sectores o lineas. Está orientado a crear reportes fáciles de interpretar.
Es multiplataforma y de uso gratuito, aunque sis capacidades estadísticas son limitadas.


\subsubsection{PSPP}
Es un proyecto de GNU como alternativa libre de SPSS (Software estadístico de IBM). Ofrece pruebas estadísticas como ANOVA
(Prueba estadística para comparar las medias de tres o más grupos), regresión (Permite modelar la relación entre variables),
correlaciones (Permite medir la fuerza y dirección de la relación lineal entre dos variables numéricas) y test no paramétricos
(Métodos estadísticos de hipótesis que no requieren que los datos provengan de una distribución específica, como la normalidad).
Tiene interfaz gráfica y lenguaje de sintaxis, soporta grandes volumenes de datos sin limitaciones artificiales y, aunque sus 
gráficos son básicos, cumple bien en análisis estadísticos en ciencias sociales y educación.

\subsubsection{JASP}
Es un programa de análisis estadístico con diseño moderno y centrado en la facilidad de uso. Usa métodos clásicos y bayesianos,
y presenta los resultados de forma clara y visual. Incluye plantillas de informes y opciones de personalización, aunque son mínimas.
Destaca en entornos educativos y académicos.

\newpage

\subsection{Librerías de programación para visualización}

\subsubsection{Pandas (Python)}
Es una librería orientada a la manipulación y análisis de datos en estructuras tipo tablas, como DataFrames. Permite realizar
operaciones estadísticas, de limpieza y transformación de datos. Incluye funciones básicas de visualización integradas, aunque
normalmente se combina con Matblotlib o Seaborn para gráficos más avanzados. Es muy popular en ciencia de datos y análisis estadístico.

\subsubsection{Matplotlib (Python)}
Es una librería estándar de visualización en Python. Soporta gráficos de líneas, barras, dispersión, histogramas, etc. Ofrece un gran
nivel de personalización en ejes, colores y estilos. Es la base sobre la que se han construido librerías más avanzadas como Seaborn.
Es muy utilizada en ciencia de datos, ingeniería y análisis estadístico.

\subsubsection{Plotly (Python/R/JS)}
Es una herramienta para generar gráficos interactivos que permite exportar visualizaciones que pueden integrarse en páginas web. Soporta
gráficos en 2D y 3D, así como mapas y dashboards. Es útil en entornos donde la interactividad es clave, como presentaciones y análisis exploratorios.

\subsubsection{R}
Es un lenguaje de programación estadístico con un gran ecosistema para análisis y visualización de datos. Su librería más destacada es ggplot2,
que implementa la 'gramática de gráficos'. Está muy orientado a análisis avanzado y visualización de datos. Es ampliamente utilizado en estadística,
bioinformática y análisis académico.

\newpage

\subsection{Herramientas online de visualización}

\subsubsection{Infogram}
Se trata de una plataforma online para crear gráficos interactivos, infografías y dashboards. Ofrece plantillas prediseñadas y soporte
para importación de datos. Se usa en entornos de comunicación visual y marketing.

\subsubsection{Piktochart}
Es una herramienta enfocada en la creación de infografías y presentaciones. Permite diseñar elementos visuales atractivos con plantillas personalizables.
Está mas orientada al diseño gráfico que al análisis de datos profundo. Es popular en marketing y educación.

\subsubsection{Flourish}
Es una aplicación online que permite crear visualizaciones animadas y narrativas. Contiene plantillas interactivas listas para usar que
se adaptan a presentaciones y storytelling con datos. Es muy empleada en periodismo de datos.

\subsubsection{RawGraphs}
Es una aplicación web de código abierto para transformar datos en gráficos vectoriales. Ofrece diagramas menos comunes (como treemaps, Sankey, etc.)
y exporta en formatos editables como SVG. Está orientada a usuarios con experiencia en diseño gráfico y visualización avanzada.

\subsubsection{Datawrapper}
Es una plataforma usada habitualmente en medios de comunicación para crear gráficos y mapas interactivos. Permite crear gráficos y mapas
interactivos de foram sencilla. Se enfoca en la claridad y en la comunicación visual efectiva.

\subsubsection{Canva}
Es una herramienta web de diseño gráfico que permite generar gráficos sencillos a partir de datos. Su punto fuerte es el aspecto visual y
la integración con otros elementos gráficos, más que en el análisis de datos. Es muy popular en marketing y redes sociales.

\newpage

\section{Probar herramientas}
En este apartado vamos a probar dos herramientas de cada tipo con un conjunto de datos sencillo y generaremos al menos un gráfico en cada caso.

\newpage

\section{Comparar herramientas}
En este apartado vamos a comparar las herramientas usando criterios como:
\begin{itemize}
    \item Facilidad de uso.
    \item Capacidades de análisis o personalización.
    \item Calidad estética de las visualizaciones.
    \item Exportación y difusión de resultados.
    \item Coste/licencia.
    \item Comunidad y documentación.
\end{itemize}

\newpage

\section{Informe comparativo}
Y finalmente, vamos a elaborar un informe comparativo (3-4 páginas) que incluya:
\begin{itemize}
    \item Tabla comparativa de las herramientas analizadas.
    \item Capturas de las visualizaciones realizadas.
    \item Reflexión final: ¿qué herramienta elegirías en estos contextos y por qué?
        \begin{itemize}
            \item Un investigador en ciencias sociales.
            \item Una empresa que quiere difundir resultados en redes sociales.
            \item Un analista de datos que trabaja en Python.
        \end{itemize}
\end{itemize}

\end{document}
