\documentclass{../../miPlantilla}

\renewcommand{\miAsignatura}{Sistemas de Big Data}
\renewcommand{\tituloTrabajo}{Eligiendo la mejor herramienta}
\renewcommand{\imagenPortada}{portada.png}

\begin{document}

\maketitle

\section{Introducción}
El ecosistema de herramientas de visualización de datos es muy amplio: desde software estadístico de código abierto,
hasta librerías de programación y herramientas online orientadas a la comunicación visual. En este informe se van a 
comparar distintas opciones y se va a elegir la más adecuada en función del contexto de uso.

\section{Clasificación de herramientas}

Las herramientas de visualización de datos se pueden agrupar en tres tipos principales: estadísticas de código abierto,
librerías de programación y herramientas online orientadas a la comunicación visual. Cada categoría tiene características
específicas que las hacen más adecuadas según el contexto de uso y el perfil del usuario.

\begin{itemize}
    \item \textbf{Herramientas estadísticas de código abierto:}  
    Este tipo de software está diseñado para realizar análisis estadísticos completos y generar visualizaciones básicas o
    avanzadas a partir de los datos. Son opciones gratuitas que permiten procesar grandes volúmenes de información y aplicar
    métodos estadísticos sin necesidad de pagar licencias comerciales. Entre las más conocidas se encuentran:  
    \begin{itemize}
        \item \textbf{SOFA Statistics:} orientada a usuarios que buscan un entorno sencillo y amigable para análisis descriptivos,
        con gráficos automáticos y exportación fácil.  
        \item \textbf{PSPP:} alternativa libre a SPSS, permite realizar análisis estadísticos complejos, incluyendo regresiones y
        pruebas de hipótesis, con gráficos básicos.  
        \item \textbf{JASP:} combina un enfoque intuitivo con análisis estadísticos avanzados; ofrece tanto resultados clásicos
        como Bayesianos y visualizaciones listas para presentación.
    \end{itemize}

    \item \textbf{Librerías de programación para visualización:}  
    Estas librerías requieren conocimientos de programación, pero ofrecen un control total sobre la forma de procesar y representar
    los datos. Son ideales para análisis reproducibles, automatización de gráficos y personalización completa. Algunas de las más
    utilizadas son:  
    \begin{itemize}
        \item \textbf{Pandas:} librería de Python orientada al manejo de datos en estructuras tipo tabla; permite generar gráficos
        básicos integrados y preparar los datos para visualización más avanzada.  
        \item \textbf{Matplotlib:} proporciona gráficos 2D altamente personalizables, desde simples líneas hasta diagramas complejos,
        siendo una de las librerías más estables y consolidadas.  
        \item \textbf{Plotly:} permite crear gráficos interactivos en Python, R o JavaScript; útil para dashboards y visualizaciones
        dinámicas que se pueden integrar en aplicaciones web.  
        \item \textbf{R (ggplot2):} librería de R basada en la gramática de gráficos, que permite construir visualizaciones
        complejas y estéticamente consistentes a partir de datos estadísticos.
    \end{itemize}

    \item \textbf{Herramientas online de visualización:}  
    Estas plataformas están diseñadas principalmente para la creación rápida de gráficos atractivos y la difusión de resultados
    en medios digitales o presentaciones. Suelen ofrecer plantillas, gráficos interactivos y opciones de exportación sencilla.
    Algunas de las más populares son:  
    \begin{itemize}
        \item \textbf{Infogram:} permite crear gráficos, mapas e infografías interactivas con plantillas listas y
        funcionalidades colaborativas.  
        \item \textbf{Piktochart:} enfocado en la comunicación visual de datos para presentaciones y redes sociales,
        con diseño intuitivo y muchas opciones de personalización.  
        \item \textbf{Flourish:} orientado a visualizaciones interactivas avanzadas, muy usado en periodismo de datos
        y dashboards online.  
        \item \textbf{RawGraphs:} herramienta libre que convierte datos tabulares en gráficos vectoriales exportables
        (SVG, PNG) de manera rápida.  
        \item \textbf{Datawrapper:} permite generar gráficos interactivos y mapas, con integración fácil para
        publicaciones web y prensa.  
        \item \textbf{Canva:} aunque es más conocida como herramienta de diseño, incluye opciones para gráficos y
        tablas, ideal para infografías atractivas y material de difusión visual.
    \end{itemize}
\end{itemize}

\newpage

\section{Prueba de herramientas}

Para las pruebas se utilizó un conjunto de datos abierto del portal \texttt{datos.gob.es}, que incluye información
básica sobre población y servicios en distintas ciudades. Se generó al menos un gráfico en cada herramienta y se
incluyen capturas representativas.

\subsection{Herramientas estadísticas de código abierto}
% \begin{itemize}
%     \item \textbf{SOFA Statistics:} gráfico de barras de población por ciudad.
%     \begin{figure}[H]
%         \centering
%         \includegraphics[width=0.7\textwidth]{capturas/sofa.png}
%         \caption{Gráfico de barras realizado en SOFA Statistics.}
%     \end{figure}

%     \item \textbf{JASP:} gráfico de pastel mostrando proporción de servicios por ciudad.
%     \begin{figure}[H]
%         \centering
%         \includegraphics[width=0.7\textwidth]{capturas/jasp.png}
%         \caption{Gráfico de pastel realizado en JASP.}
%     \end{figure}
% \end{itemize}

\subsection{Librerías de programación}
% \begin{itemize}
%     \item \textbf{Matplotlib (Python):} gráfico de líneas con evolución de población.
%     \begin{figure}[H]
%         \centering
%         \includegraphics[width=0.7\textwidth]{capturas/matplotlib.png}
%         \caption{Gráfico de líneas realizado en Matplotlib.}
%     \end{figure}

%     \item \textbf{Plotly (Python):} gráfico interactivo de dispersión.
%     \begin{figure}[H]
%         \centering
%         \includegraphics[width=0.7\textwidth]{capturas/plotly.png}
%         \caption{Gráfico de dispersión interactivo realizado en Plotly.}
%     \end{figure}
% \end{itemize}

\subsection{Herramientas online}
% \begin{itemize}
%     \item \textbf{Datawrapper:} mapa interactivo de distribución de población.
%     \begin{figure}[H]
%         \centering
%         \includegraphics[width=0.7\textwidth]{capturas/datawrapper.png}
%         \caption{Mapa interactivo realizado en Datawrapper.}
%     \end{figure}

%     \item \textbf{Canva:} infografía combinando gráfico de barras y datos textuales.
%     \begin{figure}[H]
%         \centering
%         \includegraphics[width=0.7\textwidth]{capturas/canva.png}
%         \caption{Infografía realizada en Canva.}
%     \end{figure}
% \end{itemize}

\newpage




\section{Comparativa de herramientas}



\section{Reflexión final}

\begin{itemize}
    \item \textbf{Investigador en ciencias sociales:} Probablemente prefiera \textbf{JASP} o \textbf{SOFA Statistics}, por su facilidad de uso y enfoque estadístico.
    \item \textbf{Empresa que quiere difundir resultados en redes sociales:} Herramientas como \textbf{Canva} o \textbf{Datawrapper} destacan por su estética y facilidad para compartir visualizaciones online.
    \item \textbf{Analista de datos que trabaja en Python:} Librerías como \textbf{Matplotlib} o \textbf{Plotly} son ideales, ya que permiten gran personalización y análisis avanzado, además de integración con otros procesos de datos.
\end{itemize}

\section{Conclusión}
Cada tipo de herramienta tiene fortalezas específicas según el contexto: las estadísticas de código abierto son más adecuadas para análisis científico, las librerías de programación ofrecen control y personalización, y las herramientas online facilitan la difusión visual y atractiva de los datos.

\end{document}
