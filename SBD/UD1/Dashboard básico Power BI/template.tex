\documentclass{../../../miPlantilla}

\renewcommand{\miAsignatura}{Sistemas de Big Data}
\renewcommand{\tituloTrabajo}{Dashboard básico de Power BI}
\renewcommand{\imagenPortada}{dashboard.png}

\begin{document}

\maketitle

\section{Explicando el Dashboard}

El dashboard está dividido en 3 secciones principales:

En la \textbf{superior} he puesto unos indicadores generales:

\fig{dashboard_superior.png}

\begin{itemize}
  \item Los pedidos totales.
  \item El tiempo promedio de entrega.
  \item Distancia promedio de los repartos.
  \item Tiempo medio de preparación de un pedido.
\end{itemize}

Estos KPIs ofrecen una visión general del \textbf{volumen y la eficiencia} promedio del servicio de reparto.

En la \textbf{central} he puesto unos gráficos que muestran factores que \textbf{afectan al tiempo de entrega}:

\fig{dashboard_central.png}

Tiempo de entrega según:

\begin{itemize}
  \item Condiciones climáticas.
  \item Tipo de vehículo del repartidor.
  \item Nivel de tráfico.
\end{itemize}

Y a la derecha del todo, un gráfico de líneas que muestra el \textbf{tiempo promedio de entrega dependiendo del momento del día}.

Y en la \textbf{inferior} he puesto un histograma que hace un recuento de \textbf{cúantos pedidos hay por cada franja de tiempo},
deferenciadas por 10 min, junto a un diagrama de dispersión que muestra una \textbf{relación entre el tiempo de entrega y la distancia de
cada pedido}, diferenciando por colores el vehículo de dicho pedido:

\fig{dashboard_inferior.png}

En conclusión, el dashboard permite \textbf{identificar los factores que más afectan a la eficiencia de las entregas} y muestra
cómo se distribuyen los tiempos en función de diversos factores. Esto ofrece una visión integral para poder \textbf{optimizar
las rutas, asignar vehículos correctamente y ver tiempos operativos en el servicio a domicilio}.

\end{document}
