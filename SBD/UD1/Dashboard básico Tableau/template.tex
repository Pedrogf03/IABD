\documentclass{../../../miPlantilla}

\renewcommand{\miAsignatura}{Sistemas de Big Data}
\renewcommand{\tituloTrabajo}{Dashboard básico de Power BI}
\renewcommand{\imagenPortada}{dashboard.png}

\begin{document}

\maketitle

\section{Explicando el Dashboard}

El dashboard está dividido en tres secciones principales, organizadas verticalmente. Cada bloque permite analizar distintos aspectos de la relación entre el tiempo en redes, el bienestar emocional y algunos factores demográficos.

En la \textbf{parte superior} del dashboard se presentan los indicadores promedio más relevantes:

\fig{dashboard_superior.png}

\begin{itemize}
  \item \textbf{Tiempo en pantalla promedio:} 5,5 horas diarias.
  \item \textbf{Felicidad promedio:} 8,4 sobre 10.
  \item \textbf{Estrés promedio:} 6,6 sobre 10.
  \item \textbf{Promedio de días sin redes sociales:} 3,1 días.
\end{itemize}

Estos valores ofrecen una visión general del comportamiento medio de los usuarios, sirviendo como punto de partida para el análisis posterior.

La \textbf{parte central} muestra la composición de la muestra de usuarios:

\fig{dashboard_central.png}

\begin{itemize}
  \item Un \textbf{gráfico circular} representa la proporción de usuarios por género, con predominio de hombres (248), seguido de mujeres (229) y un grupo minoritario clasificado como “Otro” (23).
  \item Un \textbf{gráfico de barras} muestra la \textbf{distribución de usuarios por edad}, con mayor concentración entre los 35 y 40 años.
\end{itemize}

Esta información contextualiza los datos, permitiendo entender mejor la población sobre la que se realiza el estudio.

En la \textbf{parte inferior} se analizan las relaciones más relevantes entre las variables del conjunto de datos:

\fig{dashboard_inferior.png}

\begin{itemize}
  \item \textbf{Plataforma vs Felicidad:} la felicidad promedio es ligeramente superior en usuarios de LinkedIn y X (Twitter).
  \item \textbf{Tiempo de pantalla vs Estrés:} se observa una tendencia positiva, indicando que a mayor tiempo frente a la pantalla, mayor nivel de estrés.
  \item \textbf{Ejercicio vs Estrés:} el estrés disminuye a medida que aumenta la frecuencia de ejercicio, siendo este efecto más notable en hombres.
\end{itemize}

Estas visualizaciones permiten identificar patrones y posibles relaciones entre el uso de redes sociales, la salud mental y los hábitos de vida.

El dashboard ofrece una visión sintética pero informativa sobre cómo el uso de redes sociales y el estilo de vida influyen en el bienestar psicológico. Su estructura por bloques facilita una lectura progresiva: primero el contexto general, luego los datos demográficos y finalmente las relaciones analíticas.

\end{document}
