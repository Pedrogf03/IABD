\documentclass{../../../miPlantilla}

\renewcommand{\miAsignatura}{Sistemas de Big Data}
\renewcommand{\tituloTrabajo}{Visualización 2}
\renewcommand{\imagenPortada}{graf3.jpeg}

\begin{document}

\maketitle

\section{Gráfico original}

\fig{graf3.jpeg}

Este gráfico de barras intenta mostrar el porcentaje de personas que se sienten agobiadas por la cantidad de noticias disponibles
en diferentes países, comparando datos de 2019 y 2024. Sin embargo, su diseño presenta múltiples problemas que dificultan enormemente
su comprensión; como la excesiva decoración provocando demasiado ruido visual, la superposición de barras crea confusión sobre qué
valor estamos observando, etc.

\newpage

\section{Mi versión}
Por los motivos que hemos visto, he creado otro gráfico que trata de transmitir la información de otra forma mas visual y sencilla:

\fig{grafico_modificado.png}

En esta versión, podemos ver una mejora clara respecto al original. Empezando por el \textbf{enfoque narrativo}, me he centrado más en el caso
concreto de \textbf{España} y la \textbf{tendencia general}, creando un relato más potente que la comparación de los países.

También se centra en una \textbf{evolución}, en lugar de puntos aislados, la \textbf{progresión} permite ver tendencias en lugar de comparaciones estáticas
y la linea de tendencia ayuda a \textbf{contextualizar los datos individuales}.

En cuanto a la \textbf{jerarquía visual}, se destaca \textbf{España como protagonista}, \textbf{Estados Unidos} y \textbf{Japón} como contrapuntos relevantes, y la tendencia
general como referencia. Esto elimina el \textbf{ruido visual del resto de paises}, que no nos ineteresan tanto.

Gracias a los ejes etiquetados, la escala consistente y fácil de seguir, el espaciado entre los elementos y la leyenda que está integrada
naturalmente, vemos el gráfico mucho más \textbf{limpio y legible}.

En conclusión, la versión original hace que el espectador tenga que \textbf{trabajar para entenderlo}; mientras que en mi versión, \textbf{el mensaje
principal se entiende de un vistazo}. No solo muestra datos, si no que \textbf{cuenta una historia sobre el creciente agobio informativo}, destacando
España. 

\end{document}
