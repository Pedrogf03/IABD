\documentclass{../../miPlantilla}

\renewcommand{\miAsignatura}{Big Data Aplicado}
\renewcommand{\tituloTrabajo}{Visualización 1}
\renewcommand{\imagenPortada}{RA2b_1.png}

\begin{document}

\maketitle

\section{Storytelling}
El gráfico cuenta una historia global: los países con \textbf{mayores ingresos} disfrutan de una \textbf{mayor esperanza de vida}.

\fig[1\linewidth]{RA2b_1.png}

\textbf{Asia} muestra grandes diferencias entre paises, con \textbf{India y China} destacando por su población y rápido desarrollo.
\textbf{África} se encuentra bajo en ambas variables, dando a entender desafíos estructurales. \textbf{Oceanía} no destaca en cuanto
a población, pero \textbf{Australia} muestra niveles altos de renta; mientras que \textbf{Europa} concentra países con altos
niveles de desarrollo y salud. \textbf{América} presenta una distribución intermedia, mientras que el norte parece estar mas equilibrado,
la tendencía del sur muestra una subida exponencial.
La narrativa visual revela inequidades globales y sugiere una \textbf{relación directa entre bienestar económico y salud poblacional}.

\newpage

\section{Analisis del gráfico}

\fig[1\linewidth]{RA2b_1.png}

\subsection{Ratio tinta/datos (Edward Tufte)}
El gráfico trata de mostrar la mayor cantidad de información posbile y reduciendo el “ruido visual”. No hay decoraciones
innecesarias: cada elemento comunica datos. Los colores, tamaños y líneas se emplean solo para representar variables (continente,
población y tendencia), lo que mantiene una alta relación tinta/datos.

\subsection{Leyes de la Gestalt}

\begin{itemize}
    \item Proximidad: Cada continente está en su propio cuadro, facilitando la comparación sin confusión visual.
    \item Semejanza: El color de cada continente permite identificar grupos de países de un vistazo.
    \item Continuidad: Las líneas de tendencia dentro de cada panel guían la mirada del observador y refuerzan la relación
            positiva que existe entre el desarrollo económico y la esperanza de vida.
\end{itemize}

\subsection{Atributos preatencionales}

\begin{itemize}
    \item Posición: El eje X (ingreso per cápita) y el eje Y (esperanza de vida) permiten percibir correlaciones sin esfuerzo cognitivo.
    \item Color: La diferencia clara de continentes permite identificar patrones regionales.
    \item Tamaño: El área del círculo refleja la población; así se capta de inmediato la magnitud relativa de cada país.
\end{itemize}

\end{document}
