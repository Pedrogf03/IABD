\documentclass{../../../miPlantilla}

\renewcommand{\miAsignatura}{Big Data Aplicado}
\renewcommand{\tituloTrabajo}{Análisis y diseño datawarehouse 1}
\renewcommand{\imagenPortada}{model.png}

\begin{document}

\maketitle

\section{Análisis del problema}
El objetivo es diseñar un almacén de datos que permita a los responsables de proyecto
y división analizar las \textbf{horas registradas por empleado y proyecto}, así como
realizar un análisis temporal por día, mes, trimestre y año.

\fig[0.5\textwidth]{1.png}

\newpage

\section{Diseño del modelo}
Para dar soporte a los requerimientos, he diseñado un modelo de tipo \textbf{estrella}.
La granularidad escogida corresponde a las \textbf{horas registradas por un empleado
en un proyecto en un día determinado}, lo que permite analizar a nivel de empleado,
proyecto, división o periodo temporal (día, mes, trimestre o año).

El modelo está formado por una \textbf{tabla de hechos} y cuatro \textbf{tablas de dimensiones}:
\fig[1\textwidth]{model.png}

En la tabla de hechos, la medida principal es el campo \textbf{Hours\_logged}, que representa una variable numérica
que se puede agregar y analizar bajo diferentes perspectivas.

Sobre las tablas de dimensiones:

\begin{itemize}
  \item La dimensión \texttt{Employee} contiene los atributos que definen a un empleado.
  \item La dimensión \texttt{Project} contiene los atributos que definen un proyecto.
  \item La dimensión \texttt{Division} dontiene los atributos que definen una división o departamento.
  \item La dimensión \texttt{Time} contiene atributos sobre la fecha, permitiendo realizar análisis por diferentes niveles de agregación temporal.
\end{itemize}

Todas las dimensiones están relacionadas directamente con la tabla de hechos, ya que ayudan a definir
el propio hecho.

\newpage

\section{Justificación del modelo}
El campo \texttt{Hours\_logged} se ha incluido en la tabla de hechos y no en la dimensión de empleados
ya que representa una \textbf{medida cuantitativa} que puede variar con el tiempo y contexto del proyecto.

Las dimensiones se limitan a información descriptiva y estática, al contrario que la tabla de hechos,
que contiene valores numéricos que pueden agregarse para analizarlos.

De esta forma, mi modelo permite analizar:

\begin{itemize}
  \item Horas totales trabajadas por cada empleado.
  \item Horas trabajadas por proyecto o prioridad.
  \item Distribución de horas por división o responsable.
  \item Evolución temporal de las horas trabajadas por mes, trimestre o año.
\end{itemize}

\end{document}
