\documentclass{../../../miPlantilla}

\renewcommand{\miAsignatura}{Big Data Aplicado}
\renewcommand{\tituloTrabajo}{Dremio}
\renewcommand{\imagenPortada}{image.png}

\begin{document}

\maketitle

% --------------------
% Actividad 8: Instalando la herramienta Dremio
% --------------------
\section{Instalando Dremio}
En esta actividad vamos a instalar y aprender a usar la herramienta Dremio. Para ello vamos a comenzar con la configuración inicial:

Creamos un usuario de sistema y el grupo asociado:

\fig{P8/1.png}
\fig{P8/2.png}

También tenemos que preparar la estructura de directorios en \path{/opt}:

\fig{P8/3.png}

Ahora vamos a descargar el software de la \href{https://download.dremio.com/community-server/}{página oficial}:

\fig{P8/4.png}

\newpage

Lo descomprimimos en la ruta que hemos preparado:

\fig{P8/5.png}
\fig{P8/6.png}

Una vez hecho, vamos a crear un enlace simbólico de la configuración en \path{/etc} y copiamos el servicio para el arranque:

\fig{P8/7.png}

Arrancamos dremio y, cuando finalice, lo configuramos para que se arranque al iniciar:

\fig{P8/8.png}
\fig{P8/9.png}

Finalmente, vamos a comprobar si podemos acceder:

\fig{P8/10.png}

Como paso extra, voy a configurar un reenvio de puerto en la máquina virtual para acceder a dremio desde el host:

\fig{P8/11.png}
\fig{P8/12.png}

\newpage

% --------------------
% Actividad 9: Trabajando con distintos formatos en dremio
% --------------------
\section{Trabajando con formatos en Dremio}
En este apartado vamos a usar un dataset de muestra y hacer operaciones con él en diferentes formatos.

Primero creamos una carpeta en dremio para guardar todos los ficheros:

\fig{P9/2.png}
\fig{P9/3.png}

Dentro de la carpeta, tenemos que subir el archivo csv de muestra:

\fig{P9/4.png}
\fig{P9/5.png}

\newpage

Al subirlo, tenemos que activar la opción para que extraiga los nombres de las columas y no los trate como un registro más:

\fig{P9/6.png}
\fig{P9/7.png}

Si entramos en el dataset, podemos hacer consultas como si de una base de datos se tratase:

\fig{P9/8.png}

\newpage

Hacemos una consulta para mostrar todos los datos y vamos a descargar este resultado en dos formatos, en \path{JSON} y en \path{PARQUET}:

\fig{P9/9.png}

Y subimos los ficheros a Dremio como hicimos con el csv inicial:

\fig{P9/10.png}

Ahora vamos a lanzar una consulta de prueba en cada uno de los datasets para comprobar que dataset la realiza más rápido:

\begin{itemize}
    \item CSV:
    \fig{P9/11.png}
    \item JSON:
    \fig{P9/12.png}
    \item PARQUET:
    \fig{P9/13.png}
\end{itemize}

Podemos revisar los detalles de los trabajos:

\fig{P9/14.png}
\fig{P9/15.png}

\newpage

Finalmente, vamos a realizar una comprobación. Queremos saber si PARQUET es autodescriptivo. Para esto vamos a cargar datos de prueba
y crear un dataset a partir de un fichero parquet.

\fig{P9/16.png}
\fig{P9/17.png}
\fig{P9/18.png}

Dentro de los datos de prueba, hay varios ficheros en distinto formato:

\fig{P9/19.png}

\newpage

Si miramos el csv, podemos ver que este no es autodescrptivo:

\fig{P9/20.png}

Sin embargo, si miramos el JSON, veremos que si lo es:

\fig{P9/21.png}

Vamos a comprobar que ocurre con un archivo PARQUET.

\fig{P9/22.png}
\fig{P9/23.png}

Como podemos observar, PARQUET si es autodescriptivo.

\newpage

% --------------------
% Actividad 10: Transformando datos en Dremio
% --------------------
\section{Transformando datos en Dremio}
En este apartado vamos a empezar a partir de un dataset y vamos a transformar los datos a nuestra elección.

Para ello, vamos a crear un espacio llamado 'Ventas':

\fig{P10/1.png}
\fig{P10/3.png}

Ahora vamos a realizar una consulta a la tabla Supermarket y guardamos el resultado como un dataset nuevo dentro del espacio que hemos creado:

\fig{P10/4.png}

\newpage

Lo guardamos como 'Ventas':

\fig{P10/5.png}
\fig{P10/6.png}

Lo primero que vamos a hacer es crear una wiki del dataset. Esto nos permite tener la documentación dentralizada, aportando contexto y 
trazabilidad a los datos.

\fig{P10/7.png}

\newpage

Añadimos una descripción al dataset:

\fig{P10/8.png}

Y ahora, cuando hagamos consultas, podemos tener la documentación siempre a mano:

\fig{P10/9.png}

\newpage

Lo siguiente que vamos a hacer es cambiar los tipos de datos. Empezaremos por cambiar a formato número (float o entero) los campos numéricos
pero que estan puestos en formato texto. Seleccionamos el campo y le damos a 'Convert data Type...'

\fig{P10/10.png}

\newpage

Dentro, ponemos el tipo de campo al que queremos convertirlo:

\fig{P10/11.png}

Y vemos un preview del resultado:

\fig{P10/12.png}

\newpage

Si aplicamos, vemos la consulta generada automáticamente para la conversión del tipo. Si repetimos el proceso con el resto de campos, 
llegamos a este resultado:

\fig{P10/13.png}

Ahora vamos a limpiar los espacios de las cadenas y capitalizar el texto:

\fig{P10/14.png}
\fig{P10/15.png}
\fig{P10/16.png}

Y, finalmente, vamos a separar por partes el ID y luego lo vamos a volver a unir:

\fig{P10/17.png}
\fig{P10/18.png}

* Aquí es importante no borrar el campo original.

\fig{P10/19.png}
\fig{P10/20.png}
\fig{P10/21.png}
\fig{P10/22.png}

La consulta final quedaría así:

\fig{P10/23.png}

Y mostraría este resultado:

\fig{P10/24.png}

\newpage

Vamos a guardarlo como un dataset en el espacio de ventas:

\fig{P10/25.png}
\fig{P10/26.png}


\end{document}