\documentclass{../../miPlantilla}

\renewcommand{\miAsignatura}{Big Data Aplicado}
\renewcommand{\tituloTrabajo}{Apache NiFi}
\renewcommand{\imagenPortada}{image.png}

\begin{document}

\maketitle

% --------------------
% Actividad 1: Instalación de Lubuntu
% --------------------
\section{Lubuntu 24.04}
En esta práctica instalaremos y configuraremos el sistema operativo \textbf{Lubuntu 24.04} en \textbf{VirtualBox}.
Para ello debemos descargar la ISO desde la \href{https://cdimage.ubuntu.com/lubuntu/releases/noble/release/lubuntu-24.04.3-desktop-amd64.iso}{página oficial}.  

Una vez descargada, creamos la máquina virtual en VirtualBox. En este caso, se le asignaron 8 GB de RAM, 4 procesadores y 50 GB de disco reservado dinámicamente.  
Es recomendable activar la opción \textbf{EFI} y habilitar el \textbf{PAE/NX} en la configuración del sistema:

\fig{P1/4.png}

Tras esto, iniciamos la máquina e instalamos el sistema de manera habitual:

\fig{P1/6.png}

Finalmente, instalamos las \textbf{Guest Additions} y tendremos la máquina lista para trabajar:

\fig{P1/7.png}

\newpage

% --------------------
% Actvidad 2: Instalación y configuración de Apache NiFi
% --------------------
\section{Preparación del entorno}
En esta sección prepararemos el entorno para instalar y configurar \textbf{Apache NiFi} en la máquina virtual con Lubuntu 24.04.  

El primer paso consiste en instalar la \textbf{JDK}:

\fig{P2/1.png}

Comprobamos que la instalación se ha realizado correctamente verificando la versión instalada:

\fig{P2/2.png}

A continuación, configuramos las variables de entorno de \textbf{Java}. Desde el directorio \textbf{home}, editamos el archivo \path{.bashrc} y añadimos las siguientes líneas al final:

\fig{P2/3.png}

\newpage

En paralelo, instalamos la versión \textbf{JDK 21} para que NiFi utilice esta versión, manteniendo la JDK 11 como predeterminada en el sistema:

\fig{P2/4.png}

Posteriormente comprobamos las rutas de instalación y configuramos \textbf{Java 11} como versión predeterminada:

\fig{P2/5.png}

\textit{Nota: seleccionar el número correspondiente a la versión 11.}

\fig{P2/6.png}

A continuación descargamos \textbf{Apache NiFi} desde la \href{https://nifi.apache.org/}{página oficial}:

\fig{P2/7.png}
\fig{P2/8.png}

Una vez descargado, lo descomprimimos:

\fig{P2/9.png}

Y lo movemos a un directorio más adecuado:

\fig{P2/10.png}

Definimos también las variables de entorno de NiFi, editando nuevamente el archivo \path{.bashrc} y añadiendo lo siguiente:

\fig{P2/11.png}

Ahora intentamos arrancar NiFi:

\fig{P2/12.png}

Observamos que aparece un error, ya que la versión 11 de Java no es compatible con esta versión de NiFi.

\newpage

Para solucionarlo, editamos el archivo \path{nifi-env.sh} e indicamos que NiFi utilice la JDK 21:

\fig{P2/13.png}
\fig{P2/14.png}

Como se aprecia, ahora NiFi arranca sin problemas. Lo detenemos con \path{nifi.sh stop} y configuramos un usuario y contraseña:

\fig{P2/16.png}

Si lo iniciamos nuevamente y accedemos a \textbf{https://localhost:8443}, llegamos al panel de inicio de sesión de NiFi, donde podemos entrar con las credenciales configuradas:

\fig{P2/17.png}

\newpage

Para facilitar la gestión y el acceso desde el host, modificamos el archivo \path{nifi.properties}, ubicado en \path{/opt/nifi/conf}, de la siguiente forma:

\fig{P2/19.png}
\fig{P2/20.png}
\fig{P2/21.png}

Dado que la máquina virtual está configurada en modo \textbf{NAT}, no es posible acceder directamente desde el exterior. Para solucionarlo, configuramos un reenvío de puertos en VirtualBox, redirigiendo el puerto 8081 de la máquina virtual al 8080 de la máquina anfitriona:

\fig{P2/23.png}
\fig{P2/24.png}

\textit{Nota: el puerto anfitrión corresponde al host, mientras que el puerto invitado corresponde al servicio de NiFi en la máquina virtual.}

\newpage

Tras configurar esto, arrancamos NiFi y ya podemos acceder fácilmente desde el host:

\fig{P2/22.png}

\newpage

% --------------------
% Actividad 3: Manejo de ficheros con Apache NiFi
% --------------------
\section{Manejo de ficheros}
A continuación aprenderemos a manejar ficheros de forma sencilla con NiFi, creando un par de flujos para generar, renombrar y mover archivos.  

Primero creamos los directorios de trabajo en la máquina:

\fig{P3/1.png}

Accedemos a NiFi y comenzamos con el primer flujo. Arrastramos un proceso al área de trabajo:

\fig[0.5\textwidth]{P3/2.png}

Seleccionamos el tipo de proceso. En este caso, \textbf{GenerateFlowFile}, que genera archivos:

\fig{P3/3.png}

\newpage

Hacemos doble clic en el proceso y lo configuramos. En la pestaña \textit{Scheduling}, establecemos que se ejecute cada 20 segundos:

\fig{P3/4.png}

En la pestaña \textit{Properties}, en el campo \textit{Custom Text}, escribimos lo siguiente:

\fig{P3/5.png}

Esto hará que cada archivo generado contenga dicho texto junto con la fecha y hora de creación.  

Luego añadimos otro proceso, \textbf{UpdateAttribute}, para modificar atributos del fichero creado:

\fig{P3/6.png}

\newpage

En sus propiedades, añadimos un nuevo atributo haciendo clic en el icono \textbf{+}:

\fig{P3/7.png}

Lo nombramos \path{filename} (atributo del nombre de archivo) y le damos el siguiente contenido:

\fig{P3/8.png}

De esta manera, el nombre del archivo estará compuesto por la fecha y hora actuales.  
Conectamos ambos procesos arrastrando la flecha desde el primero hacia el segundo:

\fig{P3/9.png}
\fig{P3/10.png}

En la configuración de la conexión seleccionamos la prioridad \textbf{FIFO (First In First Out)}:

\fig{P3/11.png}

Así obtenemos un flujo como el siguiente:

\fig{P3/12.png}

Después añadimos un proceso \textbf{PutFile}, que colocará el fichero en el directorio de entrada:

\fig{P3/13.png}

\newpage

En sus propiedades configuramos los campos \textbf{Directory} y \textbf{Conflict Resolution Strategy}:

\fig{P3/14.png}

Para validar los resultados, añadimos dos \textbf{Funnel}: uno para los ficheros exitosos (marcando \textit{Success}) y otro para los fallidos (marcando \textit{Failure}):

\begin{figure}[H]
    \centering
    \begin{minipage}{0.1\textwidth}  % Primera imagen pequeña a la izquierda
        \fig[1\linewidth]{P3/15.png}
    \end{minipage}\hfill
    \begin{minipage}{0.8\textwidth}   % Segunda imagen más grande a la derecha
        \fig[1\linewidth]{P3/16.png}
    \end{minipage}
\end{figure}


\newpage

El resultado debería ser similar a este:

\fig{P3/17.png}

Ya tenemos el primer flujo listo. Lo seleccionamos completo (con \textbf{Shift}) y lo arrancamos:

\fig{P3/18.png}

Los procesos comenzarán a ejecutarse automáticamente:

\fig{P3/19.png}

En el directorio configurado de entrada se irán generando ficheros:

\fig{P3/20.png}

\newpage

Podemos comprobar también la cola de \textit{Success}:

\fig{P3/21.png}
\fig{P3/22.png}

Vamos a parar el flujo y limpiar las colas para comenzar con el siguiente:

\fig[0.2\textwidth]{P3/23.png}

El segundo flujo consistirá en mover los archivos del directorio de entrada a uno de salida. Para ello, creamos un proceso \textbf{GetFile}:

\fig{P3/24.png}

\newpage

En sus propiedades establecemos el directorio de entrada:

\fig{P3/25.png}

Luego añadimos un proceso \textbf{PutFile}, indicando el directorio de salida, conectamos ambos procesos y añadimos \textbf{Funnels} para controlar los flujos de éxito y fallo.
El flujo final debe quedar de la siguiente manera:

\fig{P3/26.png}

Al ejecutarlo, los ficheros se mueven del directorio de entrada al de salida:

\fig{P3/27.png}

\end{document}
